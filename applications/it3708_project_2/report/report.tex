\documentclass[a4paper,9pt]{article}

\usepackage{amsmath}
\usepackage{amssymb}
\usepackage{colortbl}
\usepackage{color}
\usepackage{fullpage}
\usepackage{graphicx}
\usepackage{grffile}
\usepackage{hyperref}
\usepackage{multirow}
\usepackage{caption}
\usepackage{subcaption}
\usepackage{times}
\usepackage{url}
\usepackage{wasysym}
%\usepackage[small,compact]{titlesec}
%\usepackage[center]{caption}

\DeclareMathOperator{\avg}{avg}

\newcommand{\vvec}[1]{\ensuremath{\mathbf{#1}}}
\newcommand{\tightit}{\vspace{-5pt}}

\newcommand{\casedata}[4]{
    \begin{figure}[!h]
        \centering
        \begin{subfigure}{.5\textwidth}
          \centering
          \includegraphics[width=220pt]{#1}
        \end{subfigure}%
        \begin{subfigure}{.5\textwidth}
          \centering
          \includegraphics[width=220pt]{#2}
        \end{subfigure}
    \end{figure}

    { \scriptsize
    \begin{tabular}{c|c|c|c|c|c||c}
    \textbf{\#} & \textbf{a} & \textbf{b} & \textbf{c} & 
      \textbf{d} & \textbf{k} & \textbf{Fitness} \\ \hline
      #3
    \end{tabular} 
    \textsc{~~~~~~diameter:} #4
    \ \\
    }
}

\newenvironment{tightquote}
{ \vspace{-2pt} \begin{quote} }
{ \end{quote} \vspace{-2pt} }

\renewcommand\floatpagefraction{.9}
\renewcommand\topfraction{.9}
\renewcommand\bottomfraction{.9}
\renewcommand\textfraction{.1}   
\setcounter{totalnumber}{50}
\setcounter{topnumber}{50}
\setcounter{bottomnumber}{50}


\begin{document}

{
\centering
\Large{IT3708 Assignment 2: Evolving Spiking Neurons}  \\
\vspace{3pt}
\small{Edvard K. Karlsen} \\
}

\vspace{-2pt}

\section{Introduction}
\tightit
In this text I report on my results doing Assignment 2 for IT3708: Sub-symbolic
methods in AI. For this assignment, we should extend the EA frameworks we created
for the course's first assignment, to evolve configurations for Izhikevich's
spiking neuron model~\cite{izhikevich}.

I present my solution in Section~\ref{sec:systemoverview} (Deliverable 1), and
in a subsection~(Sec.~\ref{sec:mapping}), I discuss the classification of the
genotype-phenotype mapping used in this assignment (Deliverable 3).  Then, I
discuss my choices of EA parameter values, and my experimental setup.  in
Section~\ref{sec:experimentalsetup} (Deliverable 2). Further, I present and
discuss my results for the 12 test cases in Section~\ref{sec:results}
(Deliverable 2, continued). I address items 5 and 6 from the assignment text
in Section~\ref{sec:implications} and Section~\ref{sec:extensions}. (As is
evident, I present each deliverable quite out of order from the assignment
text. But they are all present.)

\section{System overview}
\tightit \label{sec:systemoverview}

I implemented the system within the Python framework I created for the first
homework assignment in the course, adding classes for problem-specific (i)
phenotype representation, (ii) fitness evaluation, and (iii) phenotype
development. For efficiency \footnote{I originally planned to use
meta-evolution to estimate EA parameters.}, I implemented my simulation of the
Izhikevich model -- and the SDMs -- in an optimised C++ program, which is
called from the Python code as the GA runs.
Further, I implemented a JavaScript-based front end to the simulation
procedures, to get an interface for intuitive experimentation with the
Izhikevich model.\footnote{This front end can be tested at
\url{http://folk.ntnu.no/edvardkk/izhikevich_sandbox/}.}
%During experimentation, I added real-valued vectors as a genotype choice for
%my framework, and genetic operators suitable for real-valued genotypes.

\subsection{Representation, phenotype development, and variation operators}
I experimented with two alternative setups.

First, I tested a continuous-domain solution: For
genotypes, I used five-component real vectors with each component constrained
in $[0, 1]$. I mapped into phenospace by linearly translating each
component into the permitted range of its corresponding model parameter. For
crossover, I tested both traditional one-point combination, and interpolation
of the parents' gene values. Finally, to mutate offspring, I added a random
number drawn from the Gaussian distribution at $(0, \sigma)$ with set
probability $p_m$, and subsequently clamped the result back to $[0, 1]$ when
necessary. 
My initial experiments using continuous-valued representation were
promising, but I found it very hard to find a suitable choice for $\sigma$\footnote{As
is intuitively easy to understand, a large $\sigma$ is helpful in the starting
phase of evolution, to effectively move in the search space, while a much
smaller $\sigma$ might be essential for effective exploitation near an optimum
in late iterations.}, and finally chose to abandon this approach.
%suspect that with an intelligent way to control $\sigma$ dynamically, this
%continuous-valued representation could be ideal for this problem.

Second, I tested bit vector genotypes, and finally I settled with the following
configuration: I interpret each sequence of $K=12$ bits of a $5K=60$-bit vector
as its integer value using standard binary coding, and further normalise it
into the $[0, 1]$ range and apply the translation into phenospace described for
real-valued genotypes.  To determine $K$, I tested various values using
\emph{very} large population sizes, and found that with $K=12$, precision was
high enough to code the global optimums.  For variation, I use two-point
crossover, and stochastic per-bit flipping.  \footnote{Doing this write up, I
have realised that I have a minor off-by-one bug in my binary-to-floating-point
mapper that has cut off the top $2.44 * 10^{-4}$ of each parameter's range
(with my choice $K=12$). It seems however that this minimal restriction of the
ranges has not had any practical consequences for my experiments.}

The problem text calls for text and diagrams, and I hope my ascii art
satisfies the last requirement: 
{\scriptsize
\begin{verbatim}
          [GENOTYPE: Bit vectors of length 12*5]
                        V
         <DEVELOPMENT METHOD: (i) Binary encoding,          
          (ii) scaling into parameter domains.>             
                        V
        [PHENOTYPE: 5-element real vectors, specialised]
        [with "potential" and "spike train" properties.]
                        V
   <             FITNESS EVALUATION                        >
   < Spike time, - interval, and waveform distance metrics >
                        V
            <     ADULT SELECTION           >   >>>>>   <     PARENT SELECTION           >
            < Over-production + elite sieve >   >>>>>   <  Roulette wheel w/rank scaling >
                                                                     V
                                                        <        REPRODUCTION                 >
                                                        <  Two-point x-over, per-bit mutation >
\end{verbatim}
}

\subsection{Classifying the genotype-phenotype mapping}
\tightit \label{sec:mapping}
%\begin{tightquote}
%    \footnotesize
%    \emph{Based on what you've learned about the different representational
%    forms in this course, how would you classify the genotype-phenotype
%    mapping in this exercise? Explain.  }
%\end{tightquote}
The development method is \emph{generative}, since the genes code parameters
for a development method.  

Although a bit off-question, let's also consider the
genotype classification.  The genotype I use is \emph{fixed-length linear}
(`[...] individuals that consist of fixed-length vectors of genes'
\cite{dejong}), and \emph{data oriented} (the gene data does not
\emph{directly} encode program statements or expressions). 

\subsection{Fitness evaluation}
Because the SDMs grow smaller as similarity improves, one
needs a transformation step to adapt then to our frameworks' selection
mechanisms, which are designed for maximization problems. One possibility is
to negate the fitness values, and then add a constant factor to all, so as to
make them non-negative (non-negative fitness values is an unstated requirement
for many selection mechanisms). De Jong~\cite{dejong} gives the
formula
$
\emph{fitness}(x, y) = \emph{optimality}(x) + \emph{pop\_min}(t)
$
where $\emph{pop\_min}(t)$ is the negation of the minimal fitness value at
time $t$, or alternatively, the minimum seen during the last $g$ iterations,
giving a `smoothened' transformation. 
Another possibility is taking the fitness value of some parameter configuration
with spike distance $\delta$ as $-a/\max(c, \delta)^b$, for some
well-selected constants $a$, $b$, and $c$. 

I initially used the first kind of transformation, taking \emph{pop\_min} from
the last $g=5$ iterations, but, because I settled on using tournament or rank
selection in my experiments with selection mechanisms (which I will discuss in
the next section), I realised that I could simply use the negations of the raw
fitness values, since tournament and rank selection clearly cares only about
\emph{rank} between fitness values. Thus, to transform the distance metric
values to suitable fitness values, I only negate them. 

\section{GA configuration and experimental setup}
\tightit \label{sec:experimentalsetup}
Configuring an EA for a challenging problem is a daunting task: There are many
`categorical' (`which?') and quantitative alternatives for selection
protocols, selection mechanisms, and genetic operations, and the effects of
each choice are often complex and non-linear. Further, EA experiments have a
great degree of stochasticity -- the state of the RNG may result in a rare
very bad (good) run even with very strong (bad) parameters.  Because of this
difficulty of tuning, EA researchers have put some effort in investigating
automatic configuration techniques~\cite{autoparam}.  I originally planned to
use Greffenstette's Meta-EA approach~\cite{metaea} for this
assignment, but because
implementing the code for the main neuron model and the distance metrics took
longer time than I expected, I started to fear that I did not have
enough time, and chose instead to follow the traditional route and sit for (far
too many) days trying to tune the GA by hand.

Izhikevich's model is very complex, and my impression is that the
parameter space is highly `deceptive': GA search is often lead into promising
sub-spaces that can approximate a solution to some degree, but that in reality
lays far from the sub-space containing a global optimum. Because of this, I
reasoned that, while I needed some selection pressure to promote exploitation
of promising areas, I should also let (possibly \emph{much}) less-fit
individuals search alternative areas without immediately being
thrown out. I found a working configuration by combining (i) over-production,
(ii) allowing a small amount of adults to survive in an elitist sieve, and
(iii) using a rank-based selection mechanism\footnote{Which also gives very
low-fit individuals a fair chance of mating.}.

To determine crossover and mutation rates I executed 20
random runs using $3*4=12$ `suitable' choices for each of the 12 test cases.
The results of this test run can be seen in Appendix~\ref{app:ratetest}.
Analysing the results, I found that, while some configurations were clearly
bad for \emph{some} problems, there were no definitive optimal configurations
for each problem type. I finally decided to go with $p_c = 0.6$ and $p_m =
0.05$ as these seemed to be reasonable compromises, and later on decided to
increase the mutation probability to $0.06$. 

My final configuration is as follows:

\begin{quote}
{\footnotesize
\vspace{-3pt}
\begin{description}
  \item[Adult selection strategy:] 
      Over-production, allowing 100 individuals
      to reach adulthood. Two sieves are combined: (i) one elitism sieve
      selects the 10 most fit individuals from the previous population, and
      (ii) one sieve selects the 90 most fit of the newly produced children.
      As an extra measure to prevent premature convergence to a local optimum,
      only one copy of a phenotype is allowed to reach adulthood.
  \item[Parent selection strategy:] 
      Roulette wheel selection of 170
      individuals from the full adult pool using rank scaling with $\emph{min}
      = 0.5$ and $\emph{max} = 1.8$.
  \item[Crossover operator:] Two-point crossover at random cut points. 
  \item[Crossover probability:] $0.6$.
  \item[Mutation operator:] Stochastic bit flipping. (The traditional version,
      with no knowledge of `phenotype' boundaries.) 
  \item[Mutation probability:] $0.06$ per bit.
  \item[Stopping criteria:] The evolutionary loop runs for 300 iterations.
\end{description}
}
\end{quote}

For my experiments I ran each of the 12 problem configurations five times with
different RNG seeds. I recorded the final model parameters found in each of
these 60 runs. Further, for the run in each group of five giving the most fit
individual, I plotted the means, maximums, and standard deviations at each
iteration of the run, and the best-fit individuals spike train.

Also, for each problem configuration I computed the maximum `normalised'
phenospace distance among the configuration's five solutions, to investigate
the (non)similarity among near-optimal solutions . More precisely: Given two
phenotypes with parameter vectors \vvec{a} and \vvec{b}, I scaled each
component back into $[0, 1]$ and computed the distance between the resulting
`normalised' parameter vectors. This metric ranges from 0 for solutions that
are equal, to $\sqrt{1+1+1+1+1} = \sqrt{5}$ for solutions that lay at different
outer borders of the phenospace. The metric, which I have denoted as
\emph{diameter}, gives an indication of the variation in the solution space. 


\section{Results and discussion}
\tightit \label{sec:results}

In this section I present and quickly discuss the results of the five runs for
each of the twelve test cases.

At first look, the quite extreme differences in fitness seen in the plots
might seem a bit strange. But, since I use rank based scaling, the results
are of course exactly the same as if all fitness values were divided by a
billion. However, the `extreme' fitness values creates a bit trouble for
visualisation and I have therefore plotted maximum fitness using a second, much
narrower axis (right $y$ axis).

\subsection{Spike time}
\tightit

\subsubsection{Reference train 1}
\tightit
The most-fit individuals achieve a near-perfect match with the reference
train.\footnote{Note that in the 20-run crossover/mutation-rate experiments shown in
the appendix, this case is seen to be solved to a global optimum for several
configurations.} The fitness plots for run \#1 shows that the initial move
towards feasible parameter space happens already around iteration 15, and that
subsequently a \emph{much} larger computational effort is spent gradually
improving the solution in very small steps. This pattern is common for most of
the twelve test cases.

That standard deviation and mean fitness does not change significantly after
the initial search phase, may be explained by the great differences in fitness
scores (the low-fitness individuals will dominate the calculations, and
completely hide significant fitness gain in the elitist group), and because of
the `loose' GA configuration I discussed in
Section~\ref{sec:experimentalsetup}. Even though it may seem `wasteful', I
found that having this chaotic configuration which allowed a great deal of
development among lesser-fit individuals was the surest way to consistently
increase maximum fitness towards a very strong point in phenospace.


\casedata{../data/izzy-train1.dat_spike-time_2.png}
         {../data/izzy-train1.dat_spike-time_2.dev.png}
         {
1 & 0.008 & 0.142 & -47.957 & 1.968 & 0.041 &\textbf{-0.490} \\ \hline
2 & 0.010 & 0.173 & -50.227 & 1.478 & 0.041 &\textbf{-1.058} \\ \hline
3 & 0.034 & 0.050 & -48.677 & 1.799 & 0.041 &\textbf{-1.114} \\ \hline
4 & 0.023 & 0.254 & -48.250 & 2.848 & 0.039 &\textbf{-1.929} \\ \hline
5 & 0.050 & 0.059 & -44.343 & 3.450 & 0.041 &\textbf{-2.375} \\ \hline
        }
        {0.7111}



\subsubsection{Reference train 2}
\tightit
The GA finds an almost-perfect match using this distance metric (though, cf.
the Appendix, finding a global optimum seems to be hard). The
character of development is the same as for reference train 1. 

\casedata{../data/izzy-train2.dat_spike-time_1.png}
         {../data/izzy-train2.dat_spike-time_1.dev.png}
         {
1 & 0.024 & 0.224 & -49.702 & 4.076 & 0.045 &\textbf{-0.258} \\ \hline
2 & 0.037 & 0.058 & -49.189 & 4.081 & 0.045 &\textbf{-0.432} \\ \hline
3 & 0.027 & 0.194 & -48.970 & 5.065 & 0.046 &\textbf{-1.202} \\ \hline
4 & 0.037 & 0.016 & -45.381 & 7.307 & 0.049 &\textbf{-3.393} \\ \hline
5 & 0.030 & 0.226 & -54.878 & 8.419 & 0.050 &\textbf{-4.174} \\ \hline
         }
         {0.7948}




\subsubsection{Reference train 3}
\tightit
The GA finds an almost-perfect match using this distance metric. The character
of development is the same as for the previous reference trains.

\casedata{../data/izzy-train3.dat_spike-time_2.png}
         {../data/izzy-train3.dat_spike-time_2.dev.png}
         {
1 & 0.044 & 0.246 & -39.363 & 5.357 & 0.040 &\textbf{-0.382} \\ \hline
2 & 0.050 & 0.248 & -36.226 & 6.916 & 0.039 &\textbf{-1.372} \\ \hline
3 & 0.063 & 0.195 & -34.541 & 7.685 & 0.040 &\textbf{-1.516} \\ \hline
4 & 0.048 & 0.117 & -45.686 & 2.575 & 0.041 &\textbf{-2.611} \\ \hline
5 & 0.097 & 0.074 & -34.504 & 7.612 & 0.041 &\textbf{-3.263} \\ \hline
         }
         {0.6975}


\subsubsection{Reference train 4}
\tightit
The GA finds a perfect match using this distance metric. My impression is that
this is by far the `easiest' reference train to evolve towards (this is also
evident in the appendix). The perfect match is found around generation 100,
after which no useful computation happens. Though, we observe again that the
solution does not take over the population, and that there still is activity
among lesser-fit individuals.

\casedata{../data/izzy-train4.dat_spike-time_2.png}
         {../data/izzy-train4.dat_spike-time_2.dev.png}
         {
1 & 0.003 & 0.016 & -58.821 & 9.869 & 0.079 &\textbf{-0.000} \\ \hline
2 & 0.003 & 0.202 & -62.117 & 9.833 & 0.079 &\textbf{-0.100} \\ \hline
3 & 0.003 & 0.276 & -53.022 & 9.384 & 0.076 &\textbf{-0.173} \\ \hline
4 & 0.003 & 0.209 & -46.443 & 8.047 & 0.068 &\textbf{-0.200} \\ \hline
5 & 0.003 & 0.080 & -46.650 & 8.356 & 0.070 &\textbf{-0.224} \\ \hline
         }
         {0.9031}


\subsection{Spike Interval}
\tightit
Analysing the spike interval results, it is important to remember that the
measure does not care about the locations of spike points, but only the
distances between them. Therefore, we often observe `shifted' solutions, that
could be `pushed into place' to match up with the reference trains.

\subsubsection{Reference train 1}
\tightit
The plot of the best individual is skewed all the way to the left, but the
interval distances clearly match almost perfectly. It seems that, even though
the trains start at different time steps, evolution has found the same
spike-section `shape' as in the reference train. Note that in the best run
initial development is very rapid; a solution in the $[-10, -5]$ range is
found already around the fifth iteration(!)\footnote{Observing the GA as it
runs, this is the point where `the correct shape' has been found.}, and the
best solution appears around iteration 50.

\casedata{../data/izzy-train1.dat_spike-interval_3.png}
         {../data/izzy-train1.dat_spike-interval_3.dev.png}
         {
1 & 0.001 & 0.036 & -77.522 & 8.173 & 0.056 &\textbf{-0.447} \\ \hline
2 & 0.001 & 0.074 & -79.133 & 8.313 & 0.056 &\textbf{-0.490} \\ \hline
3 & 0.003 & 0.061 & -48.726 & 2.983 & 0.045 &\textbf{-0.693} \\ \hline
4 & 0.001 & 0.022 & -77.729 & 8.666 & 0.057 &\textbf{-0.721} \\ \hline
5 & 0.001 & 0.012 & -79.695 & 8.651 & 0.057 &\textbf{-0.721} \\ \hline
         }
         {0.8604}


\subsubsection{Reference train 2}
\tightit
Also here, the plot of the best individual is left-skewed, and we observe that
if it was shifted right, it would match up quite good with the reference
train. Convergence is rapid.


\casedata{../data/izzy-train2.dat_spike-interval_2.png}
         {../data/izzy-train2.dat_spike-interval_2.dev.png}
         {
1 & 0.033 & 0.091 & -47.639 & 5.512 & 0.047 &\textbf{-0.521} \\ \hline
2 & 0.025 & 0.246 & -46.479 & 6.317 & 0.048 &\textbf{-0.742} \\ \hline
3 & 0.026 & 0.281 & -42.512 & 8.460 & 0.051 &\textbf{-1.137} \\ \hline
4 & 0.024 & 0.259 & -37.227 & 9.995 & 0.056 &\textbf{-3.203} \\ \hline
5 & 0.129 & 0.012 & -47.749 & 7.160 & 0.044 &\textbf{-4.160} \\ \hline
         }
         {1.0715}


\subsubsection{Reference train 3}
\tightit
The plot of the best individual is somewhat left-skewed, but it matches up
nicely with the reference plot (even the potential `shape' looks correct).
Here, fitness development moves significantly slower. From experiments, I have
seen that the GA often struggles with finding the parameter subspace with `two
small groups of spikes'. In the best-run plot, this `spike train type' is
found around iteration 60, and from there it uses about 60-70 additional
iterations to find the best solution.

\casedata{../data/izzy-train3.dat_spike-interval_4.png}
         {../data/izzy-train3.dat_spike-interval_4.dev.png}
         {
1 & 0.025 & 0.296 & -40.596 & 4.284 & 0.041 &\textbf{-0.186} \\ \hline
2 & 0.074 & 0.142 & -37.544 & 6.176 & 0.040 &\textbf{-0.204} \\ \hline
3 & 0.051 & 0.155 & -36.030 & 6.331 & 0.041 &\textbf{-0.433} \\ \hline
4 & 0.040 & 0.254 & -34.089 & 7.307 & 0.041 &\textbf{-0.661} \\ \hline
5 & 0.048 & 0.151 & -35.359 & 6.522 & 0.041 &\textbf{-0.821} \\ \hline
         }
         {0.6182}

\clearpage

\subsubsection{Reference train 4}
\tightit
As for the spike time metric, the GA solves this problem configuration
perfectly. Interestingly, the optimal solution is not the same as for the
spike time metric, even though the two spike trains look exactly similar (at
least I can not see any visible difference\ldots). 

\casedata{../data/izzy-train4.dat_spike-interval_4.png}
         {../data/izzy-train4.dat_spike-interval_4.dev.png}
         {
1 & 0.003 & 0.249 & -66.011 & 9.942 & 0.080 &\textbf{-0.000} \\ \hline
2 & 0.003 & 0.211 & -59.407 & 9.773 & 0.078 &\textbf{-0.000} \\ \hline
3 & 0.003 & 0.121 & -45.845 & 7.718 & 0.066 &\textbf{-0.100} \\ \hline
4 & 0.003 & 0.159 & -56.489 & 9.265 & 0.075 &\textbf{-0.141} \\ \hline
5 & 0.003 & 0.090 & -48.762 & 8.666 & 0.071 &\textbf{-0.200} \\ \hline
         }
         {0.6602}


\subsection{Waveform}
\tightit
The waveform metric is interesting. It does not care more about spikes than
any other points of the output potentials, so it will often `drop' a few
spikes to better fit larger parts of a neuron's potential plot. 

\subsubsection{Reference train 1}
\tightit
Evolution finds a near perfect match to this reference train. Evolution
stagnates at around the 155th iteration. Interestingly, the diameter metric is
significantly smaller for this metric than for the two others, indicating that
the solution set is `closer'.

\casedata{../data/izzy-train1.dat_waveform_3.png}
         {../data/izzy-train1.dat_waveform_3.dev.png}
         {
1 & 0.016 & 0.118 & -49.116 & 1.792 & 0.041 &\textbf{-0.159} \\ \hline
2 & 0.074 & 0.050 & -48.835 & 3.419 & 0.041 &\textbf{-0.277} \\ \hline
3 & 0.022 & 0.141 & -47.993 & 3.087 & 0.040 &\textbf{-0.278} \\ \hline
4 & 0.086 & 0.078 & -52.205 & 3.370 & 0.040 &\textbf{-0.291} \\ \hline
5 & 0.037 & 0.075 & -55.012 & 5.782 & 0.041 &\textbf{-0.336} \\ \hline
         }
         {0.4591}


\subsubsection{Reference train 2}
\tightit
Using the waveform metric it is very hard to get a one-to-one match for this
reference train (although I have seen it in rare test runs). 

\casedata{../data/izzy-train2.dat_waveform_4.png}
         {../data/izzy-train2.dat_waveform_4.dev.png}
         {
1 & 0.012 & 0.063 & -48.750 & 6.993 & 0.054 &\textbf{-0.427} \\ \hline
2 & 0.010 & 0.271 & -56.685 & 8.576 & 0.057 &\textbf{-0.428} \\ \hline
3 & 0.005 & 0.085 & -73.335 & 8.765 & 0.060 &\textbf{-0.453} \\ \hline
4 & 0.003 & 0.204 & -69.001 & 8.369 & 0.059 &\textbf{-0.459} \\ \hline
5 & 0.001 & 0.234 & -73.738 & 8.550 & 0.060 &\textbf{-0.471} \\ \hline
         }
         {0.7924}



\subsubsection{Reference train 3}
\tightit
Also here, the waveform metric has problems finding a very close match.

\casedata{../data/izzy-train3.dat_waveform_3.png}
         {../data/izzy-train3.dat_waveform_3.dev.png}
         {
1 & 0.061 & 0.144 & -41.499 & 6.633 & 0.040 &\textbf{-0.343} \\ \hline
2 & 0.054 & 0.134 & -44.478 & 4.726 & 0.040 &\textbf{-0.365} \\ \hline
3 & 0.057 & 0.253 & -33.809 & 7.071 & 0.039 &\textbf{-0.385} \\ \hline
4 & 0.104 & 0.048 & -42.512 & 5.582 & 0.041 &\textbf{-0.409} \\ \hline
5 & 0.023 & 0.188 & -45.540 & 4.219 & 0.041 &\textbf{-0.411} \\ \hline
         }
         {0.7807}


\subsubsection{Reference train 4}
\tightit
As noted earlier, reference train 4 seems to be the `easiest' and also with
the waveform metric, the GA manages to find a near-perfect match. 

\casedata{../data/izzy-train4.dat_waveform_2.png}
         {../data/izzy-train4.dat_waveform_2.dev.png}
         {
1 & 0.001 & 0.152 & -64.766 & 9.524 & 0.078 &\textbf{-0.169} \\ \hline
2 & 0.003 & 0.069 & -60.359 & 9.961 & 0.080 &\textbf{-0.169} \\ \hline
3 & 0.001 & 0.120 & -61.250 & 9.338 & 0.077 &\textbf{-0.173} \\ \hline
4 & 0.001 & 0.216 & -55.000 & 9.094 & 0.075 &\textbf{-0.187} \\ \hline
5 & 0.001 & 0.295 & -59.358 & 9.125 & 0.076 &\textbf{-0.200} \\ \hline
         }
         {0.7835}


\section{Practical implications}
\tightit \label{sec:implications}
%\begin{tightquote}
%    \footnotesize
%    \emph{Discuss the practical implications of the tool that you have built.
%    How might a computational neuroscientist use it?  }
%\end{tightquote}
%\noindent
Given a list of potential values, expressing a spike train recorded from, say,
a human or another animal, my system can effectively search for close-matching
configurations of Izhikevich's model (if such exist), with respect to three
different distance metrics.
After fitting a suitable model of some real-world spike train, a computational
neuroscientist can run large scale simulations (for example using a vast array
of spiking neurons in a network), or run automatic tools to investigate how
the behaviour of the observed neuron correspond to other neurons (cf.
Section~3 of the assignment text).

Thus, I will primarily say that my system can assist computational
neuroscientists with intelligently transforming `raw' data (for instance, from
a dataset) into a suitable model domain for further experimentation.

\section{Extending the tool}
\tightit \label{sec:extensions}
To ponder about possible extensions, it is helpful to generalise and ask what
our program really does: In abstract terms, we finding constant parameters for
a function (in our case given by a set of update equations) so that its output
fits a given data set (with respect to some distance metric). One idea wrap the
GA in a program that gives the user control over all these aspects:
{ \footnotesize
\begin{description}
    \item[Function] The user could input the function/system equations to
        tune, in a `human-readable' form similar to the one used at
        Wolfram$|$Alpha\footnote{\url{http://www.wolframalpha.com/}}. (Or
        alternatively, in a tiny LISP-style language. Though, that could
        require more proficiency from the users.) 
    \item[Variables] The user would choose which variables from the update
        equations that should be optimised through evolution, and which
        variables that should either be held constant, or be specified in
        another way (such as a time variable). For the variables to be
        included in the search space, the user would have to specify each
        variable's domain.
    \item[Distance metric] The program could come with a wide range of
        general-purpose distance metrics, such as the waveform DSM we used in
        this assignment. Ideally, the program should include a variety of DSMs
        that was suitable for the great majority of problems; but, in
        addition, users could have the option of adding custom distance
        metrics by writing tiny extension snippets in a suitable DSL.
\end{description}
}
In addition, there should be a mechanism allowing the user to control GA
parameters (although, sensible defaults would also be useful).  I have made an
attempt at illustrating the model component of such a program with the
following ascii art.
{ \scriptsize
\begin{verbatim}
[Generic GENOTYPE, MK-length bit vector             ]
[for M variables to be optimised, and K bits per var]
           |
<Development method translating into >       [DISTANCE METRIC         ]
<the user specified variable domains.>       [(Either system-provided ]
           |                                 [ or programmed by user) ]
           V                                          |
[PHENOTYPE corresponding to     ]                     V
[fn/system vars to be optimised ]     --->   <Fitness function>
                                                 ^     ^
                                   /-------------|     |
                   [REFERENCE DATA SET  ]    [FUNCTION TO BE FITTED,
                   [to be fitted against]    [provided by user       ]
\end{verbatim}
}
Another important part is that the program should include the live plotting
routines. Seeing the graph being fitted seems to often be just as interesting
and helpful as inspecting the final result!

One (somewhat contrived) example use case would be a simple physics
experiment: A student could record the displacement of, say, some projectile
being shot against a rugged surface, and then use the system to fit various
displacement formulas (for instance, with various models of friction) against
the recorded trail. 

\bibliographystyle{abbrv}   
\bibliography{refs}


\appendix
\tightit \label{app:ratetest}

\section{Experimental results for crossover and mutation rates}

{ \scriptsize \setlength{\tabcolsep}{4pt} \renewcommand*\arraystretch{1.2} 
\begin{tabular}{c|c|c|c||c|c|c|c||c|c|c|c||c|c|c|c}

\multicolumn{16}{c}{ \normalsize \textsc{Spike Time}} \\ 
\multicolumn{16}{c}{ } \\
\multicolumn{4}{c}{ \textbf{izzy-train1.dat}} &    \multicolumn{4}{c}{ \textbf{izzy-train2.dat}} &    \multicolumn{4}{c}{ \textbf{izzy-train3.dat}} &    \multicolumn{4}{c}{ \textbf{izzy-train4.dat}} \\ \hline
$p_c$ & $p_m$ & $f_{\max}$ & $f_{\avg}$  &    $p_c$ & $p_m$ & $f_{\max}$ & $f_{\avg}$  &    $p_c$ & $p_m$ & $f_{\max}$ & $f_{\avg}$  &    $p_c$ & $p_m$ & $f_{\max}$ & $f_{\avg}$  \\ \hline\hline
0.60 & 0.02 & -0.92 & -6.97 &    0.60 & 0.02 & -0.27 & -3.70 &    0.60 & 0.02 & -0.63 & -9.93 &    0.60 & 0.02 & {\cellcolor{green}-0.00} & -0.86 \\ \hline
0.60 & 0.03 & {\cellcolor{green}-0.00} & -2.95 &    0.60 & 0.03 & -1.26 & -3.04 &    0.60 & 0.03 & -0.49 & -4.72 &    0.60 & 0.03 & {\cellcolor{green}-0.00} & -0.30 \\ \hline
0.60 & 0.05 & -0.28 & -3.14 &    0.60 & 0.05 & -0.32 & {\cellcolor{yellow}-2.64} &    0.60 & 0.05 & -0.32 & -2.74 &    0.60 & 0.05 & {\cellcolor{green}-0.00} & {\cellcolor{yellow}-0.18} \\ \hline
0.60 & 0.10 & -0.49 & -3.91 &    0.60 & 0.10 & -0.49 & -3.29 &    0.60 & 0.10 & -0.38 & -3.42 &    0.60 & 0.10 & {\cellcolor{green}-0.00} & -0.51 \\ \hline\hline
0.70 & 0.02 & -0.20 & -10.26 &    0.70 & 0.02 & -0.38 & -2.97 &    0.70 & 0.02 & -0.77 & -12.26 &    0.70 & 0.02 & {\cellcolor{green}-0.00} & -0.59 \\ \hline
0.70 & 0.03 & {\cellcolor{green}-0.00} & -3.30 &    0.70 & 0.03 & -1.86 & -3.84 &    0.70 & 0.03 & -0.56 & -3.55 &    0.70 & 0.03 & {\cellcolor{green}-0.00} & -0.24 \\ \hline
0.70 & 0.05 & {\cellcolor{green}-0.00} & {\cellcolor{yellow}-2.14} &    0.70 & 0.05 & {\cellcolor{green}-0.20} & -3.03 &    0.70 & 0.05 & {\cellcolor{green}-0.08} & -2.76 &    0.70 & 0.05 & {\cellcolor{green}-0.00} & -0.23 \\ \hline
0.70 & 0.10 & -0.98 & -4.28 &    0.70 & 0.10 & -1.36 & -3.55 &    0.70 & 0.10 & -0.46 & -2.73 &    0.70 & 0.10 & -0.14 & -0.43 \\ \hline\hline
0.80 & 0.02 & {\cellcolor{green}-0.00} & -18.84 &    0.80 & 0.02 & -0.26 & -3.06 &    0.80 & 0.02 & -0.31 & -12.63 &    0.80 & 0.02 & {\cellcolor{green}-0.00} & -0.44 \\ \hline
0.80 & 0.03 & {\cellcolor{green}-0.00} & -2.30 &    0.80 & 0.03 & -1.18 & -3.06 &    0.80 & 0.03 & -0.17 & -6.69 &    0.80 & 0.03 & {\cellcolor{green}-0.00} & -0.81 \\ \hline
0.80 & 0.05 & -0.45 & -4.03 &    0.80 & 0.05 & -0.67 & -3.74 &    0.80 & 0.05 & -0.35 & {\cellcolor{yellow}-2.47} &    0.80 & 0.05 & {\cellcolor{green}-0.00} & -0.26 \\ \hline
0.80 & 0.10 & -0.96 & -4.63 &    0.80 & 0.10 & -0.65 & -3.55 &    0.80 & 0.10 & -0.92 & -3.45 &    0.80 & 0.10 & -0.10 & -0.40 \\ \hline\hline


\multicolumn{16}{c}{ } \\\multicolumn{16}{c}{ } \\
\multicolumn{16}{c}{ \normalsize \textsc{Spike Interval}} \\ 
\multicolumn{16}{c}{ } \\
\multicolumn{4}{c}{ \textbf{izzy-train1.dat}} &    \multicolumn{4}{c}{ \textbf{izzy-train2.dat}} &    \multicolumn{4}{c}{ \textbf{izzy-train3.dat}} &    \multicolumn{4}{c}{ \textbf{izzy-train4.dat}} \\ \hline
$p_c$ & $p_m$ & $f_{\max}$ & $f_{\avg}$  &    $p_c$ & $p_m$ & $f_{\max}$ & $f_{\avg}$  &    $p_c$ & $p_m$ & $f_{\max}$ & $f_{\avg}$  &    $p_c$ & $p_m$ & $f_{\max}$ & $f_{\avg}$  \\ \hline\hline
0.60 & 0.02 & {\cellcolor{green}-0.00} & -0.67 &    0.60 & 0.02 & -0.41 & -1.51 &    0.60 & 0.02 & -0.12 & -2.08 &    0.60 & 0.02 & {\cellcolor{green}-0.00} & -0.48 \\ \hline
0.60 & 0.03 & {\cellcolor{green}-0.00} & -0.47 &    0.60 & 0.03 & {\cellcolor{green}-0.16} & -1.28 &    0.60 & 0.03 & -0.12 & -0.56 &    0.60 & 0.03 & {\cellcolor{green}-0.00} & -0.09 \\ \hline
0.60 & 0.05 & {\cellcolor{green}-0.00} & -0.54 &    0.60 & 0.05 & -0.31 & -1.14 &    0.60 & 0.05 & -0.12 & -0.57 &    0.60 & 0.05 & {\cellcolor{green}-0.00} & -0.08 \\ \hline
0.60 & 0.10 & {\cellcolor{green}-0.00} & -0.57 &    0.60 & 0.10 & -0.35 & -0.98 &    0.60 & 0.10 & -0.12 & -0.67 &    0.60 & 0.10 & {\cellcolor{green}-0.00} & -0.20 \\ \hline\hline
0.70 & 0.02 & {\cellcolor{green}-0.00} & -0.74 &    0.70 & 0.02 & -0.67 & -1.56 &    0.70 & 0.02 & -0.08 & -2.93 &    0.70 & 0.02 & {\cellcolor{green}-0.00} & -0.41 \\ \hline
0.70 & 0.03 & -0.20 & -0.93 &    0.70 & 0.03 & -0.21 & -1.04 &    0.70 & 0.03 & {\cellcolor{green}-0.00} & -2.76 &    0.70 & 0.03 & {\cellcolor{green}-0.00} & -0.13 \\ \hline
0.70 & 0.05 & {\cellcolor{green}-0.00} & -0.47 &    0.70 & 0.05 & -0.18 & -0.96 &    0.70 & 0.05 & {\cellcolor{green}-0.00} & -0.50 &    0.70 & 0.05 & {\cellcolor{green}-0.00} & {\cellcolor{yellow}-0.07} \\ \hline
0.70 & 0.10 & {\cellcolor{green}-0.00} & -0.55 &    0.70 & 0.10 & -0.42 & -0.96 &    0.70 & 0.10 & -0.14 & -0.68 &    0.70 & 0.10 & {\cellcolor{green}-0.00} & -0.25 \\ \hline\hline
0.80 & 0.02 & {\cellcolor{green}-0.00} & -1.39 &    0.80 & 0.02 & -0.43 & -2.08 &    0.80 & 0.02 & {\cellcolor{green}-0.00} & -0.89 &    0.80 & 0.02 & {\cellcolor{green}-0.00} & -0.75 \\ \hline
0.80 & 0.03 & {\cellcolor{green}-0.00} & -0.52 &    0.80 & 0.03 & -0.31 & -1.12 &    0.80 & 0.03 & -0.12 & -0.47 &    0.80 & 0.03 & {\cellcolor{green}-0.00} & -0.17 \\ \hline
0.80 & 0.05 & {\cellcolor{green}-0.00} & {\cellcolor{yellow}-0.38} &    0.80 & 0.05 & -0.18 & -1.03 &    0.80 & 0.05 & -0.12 & {\cellcolor{yellow}-0.45} &    0.80 & 0.05 & {\cellcolor{green}-0.00} & -0.09 \\ \hline
0.80 & 0.10 & {\cellcolor{green}-0.00} & -0.63 &    0.80 & 0.10 & -0.31 & {\cellcolor{yellow}-0.82} &    0.80 & 0.10 & -0.08 & -0.51 &    0.80 & 0.10 & {\cellcolor{green}-0.00} & -0.24 \\ \hline\hline


\multicolumn{16}{c}{ } \\\multicolumn{16}{c}{ } \\
\multicolumn{16}{c}{ \normalsize \textsc{Waveform}} \\ 
\multicolumn{16}{c}{ } \\
\multicolumn{4}{c}{ \textbf{izzy-train1.dat}} &    \multicolumn{4}{c}{ \textbf{izzy-train2.dat}} &    \multicolumn{4}{c}{ \textbf{izzy-train3.dat}} &    \multicolumn{4}{c}{ \textbf{izzy-train4.dat}} \\ \hline
$p_c$ & $p_m$ & $f_{\max}$ & $f_{\avg}$  &    $p_c$ & $p_m$ & $f_{\max}$ & $f_{\avg}$  &    $p_c$ & $p_m$ & $f_{\max}$ & $f_{\avg}$  &    $p_c$ & $p_m$ & $f_{\max}$ & $f_{\avg}$  \\ \hline\hline
0.60 & 0.02 & -0.16 & -0.26 &    0.60 & 0.02 & -0.43 & -0.47 &    0.60 & 0.02 & -0.22 & -0.35 &    0.60 & 0.02 & -0.08 & -0.17 \\ \hline
0.60 & 0.03 & -0.17 & -0.25 &    0.60 & 0.03 & -0.40 & {\cellcolor{yellow}-0.46} &    0.60 & 0.03 & -0.20 & -0.33 &    0.60 & 0.03 & -0.08 & -0.15 \\ \hline
0.60 & 0.05 & -0.08 & -0.24 &    0.60 & 0.05 & -0.43 & -0.47 &    0.60 & 0.05 & -0.22 & -0.34 &    0.60 & 0.05 & -0.07 & -0.16 \\ \hline
0.60 & 0.10 & -0.24 & -0.30 &    0.60 & 0.10 & -0.43 & -0.48 &    0.60 & 0.10 & -0.26 & -0.37 &    0.60 & 0.10 & -0.09 & -0.19 \\ \hline\hline
0.70 & 0.02 & -0.08 & -0.24 &    0.70 & 0.02 & -0.40 & -0.47 &    0.70 & 0.02 & -0.22 & -0.35 &    0.70 & 0.02 & -0.08 & -0.16 \\ \hline
0.70 & 0.03 & -0.07 & -0.23 &    0.70 & 0.03 & {\cellcolor{green}-0.40} & -0.47 &    0.70 & 0.03 & -0.22 & -0.35 &    0.70 & 0.03 & -0.08 & {\cellcolor{yellow}-0.14} \\ \hline
0.70 & 0.05 & -0.07 & -0.23 &    0.70 & 0.05 & -0.41 & -0.47 &    0.70 & 0.05 & -0.21 & -0.34 &    0.70 & 0.05 & -0.08 & -0.16 \\ \hline
0.70 & 0.10 & -0.23 & -0.30 &    0.70 & 0.10 & -0.44 & -0.48 &    0.70 & 0.10 & -0.32 & -0.39 &    0.70 & 0.10 & -0.13 & -0.19 \\ \hline\hline
0.80 & 0.02 & -0.06 & {\cellcolor{yellow}-0.22} &    0.80 & 0.02 & -0.45 & -0.48 &    0.80 & 0.02 & -0.22 & -0.36 &    0.80 & 0.02 & -0.08 & -0.16 \\ \hline
0.80 & 0.03 & -0.13 & -0.24 &    0.80 & 0.03 & -0.41 & -0.47 &    0.80 & 0.03 & {\cellcolor{green}-0.14} & {\cellcolor{yellow}-0.33} &    0.80 & 0.03 & -0.08 & -0.14 \\ \hline
0.80 & 0.05 & {\cellcolor{green}-0.06} & -0.23 &    0.80 & 0.05 & -0.46 & -0.48 &    0.80 & 0.05 & -0.24 & -0.35 &    0.80 & 0.05 & {\cellcolor{green}-0.07} & -0.14 \\ \hline
0.80 & 0.10 & -0.23 & -0.28 &    0.80 & 0.10 & -0.42 & -0.48 &    0.80 & 0.10 & -0.28 & -0.38 &    0.80 & 0.10 & -0.13 & -0.18 \\ \hline\hline

\end{tabular}
}





\end{document}
